\documentclass{sigchi}

% Use this command to override the default ACM copyright statement
% (e.g. for preprints).  Consult the conference website for the
% camera-ready copyright statement.


%% BEGIN -- OVERRIDE THE DEFAULT COPYRIGHT STRIP
\toappear{
    This work is licensed under the Creative Commons Attribution-ShareAlike
    4.0 International License. To view a copy of this license, visit\\
    {\url{http://creativecommons.org/licenses/by-sa/4.0/}}.
    {Copyright \copyright~2015 Authors}}
%% END -- OVERRIDE THE DEFAULT COPYRIGHT STRIP


% Arabic page numbers for submission.  Remove this line to eliminate
% page numbers for the camera ready copy

%\pagenumbering{arabic}

% Load basic packages
\usepackage{balance}  % to better equalize the last page
\usepackage{graphics} % for EPS, load graphicx instead
%\usepackage[T1]{fontenc}
\usepackage{txfonts}
\usepackage{times}    % comment if you want LaTeX's default font
\usepackage[pdftex]{hyperref}
% \usepackage{url}      % llt: nicely formatted URLs
\usepackage{color}
\usepackage{textcomp}
\usepackage{booktabs}
\usepackage{ccicons}
\usepackage{todonotes}

% llt: Define a global style for URLs, rather that the default one
\makeatletter
\def\url@leostyle{%
  \@ifundefined{selectfont}{\def\UrlFont{\sf}}{\def\UrlFont{\small\bf\ttfamily}}}
\makeatother
\urlstyle{leo}

% To make various LaTeX processors do the right thing with page size.
\def\pprw{8.5in}
\def\pprh{11in}
\special{papersize=\pprw,\pprh}
\setlength{\paperwidth}{\pprw}
\setlength{\paperheight}{\pprh}
\setlength{\pdfpagewidth}{\pprw}
\setlength{\pdfpageheight}{\pprh}

% Make sure hyperref comes last of your loaded packages, to give it a
% fighting chance of not being over-written, since its job is to
% redefine many LaTeX commands.
\definecolor{linkColor}{RGB}{6,125,233}
\hypersetup{%
    pdftitle={document},
    pdfauthor={},
    pdfkeywords={},
    bookmarksnumbered,
    pdfstartview={FitH},
    colorlinks,
    citecolor=black,
    filecolor=black,
    linkcolor=black,
    urlcolor=linkColor,
    breaklinks=true,
}

% create a shortcut to typeset table headings
% \newcommand\tabhead[1]{\small\textbf{#1}}

% End of preamble. Here it comes the document.
\begin{document}

\title{Extending StackOverflow® Gamification Using Social Media}

\numberofauthors{5}
\author{%
    \alignauthor{Tim Baker\\
        \affaddr{University of Victoria}\\
        \affaddr{Victoria, BC, Canada}\\
        \email{timbaker@uvic.ca}}\\
    \alignauthor{Alastair Beaumont\\
        \affaddr{University of Victoria}\\
        \affaddr{Victoria, BC, Canada}\\
        \email{alastair@uvic.ca}}\\
    \alignauthor{Nitin Goyal\\
        \affaddr{University of Victoria}\\
        \affaddr{Victoria, BC, Canada}\\
        \email{ngoyal@uvic.ca}}\\
    \alignauthor{Arturo Reyes Lopez\\
        \affaddr{University of Victoria}\\
        \affaddr{Victoria, BC, Canada}\\
        \email{areyeslo@uvic.ca}}\\
    \alignauthor{Richard B. Wagner\\
        \affaddr{University of Victoria}\\
        \affaddr{Victoria, BC, Canada}\\
        \email{rbwagner@uvic.ca}}\\
}

\maketitle

\begin{abstract}
StackOverflow is a popular Q\&A community, with thousands of users asking and solving each others' programming related questions. The StackOverflow (Hereafter referred to as "SO") community uses elements of badges and reputation points to
practice gamification. Our motivation for this project is to identify the
effectiveness of gamification and whether or not it can be improved further. This project
will address issues and recommendations gathered from selected SO users to
improve/extend gamification elements. The first group of users will be selected
from the SO community based on reputation levels and the second group will be UVic
StackOverflow users. There is a definite need of improving gamification in
StackOverflow.
\end{abstract}

\keywords{Gamification, StackOverflow (SO)}

\category{H.5.m.}{Information Interfaces and Presentation
  (e.g. HCI)}{Miscellaneous} \category{See
  \url{http://acm.org/about/class/1998/} for the full list of ACM
  classifiers. This section is required.}{}{}

\section{Introduction}
In Social Media, the use of game design and game elements, well known by the term gamification, has sought to increase generation of content by the users. Badges, privileges and reputation points are being used to recognize contributions of users on the sites and generally these elements seem to be valued by users who compete for the prize. Several studies have shown awareness of other participants and competence during the game experiences\cite{Rughinis}. Other studies have shown that gamification can increase engagement in activities that involve a community \cite{Marder}. However, the inclusion of gamification itself does not guarantee engagement or motivation and it depends on the rules to earn badges and privileges  according to the context\cite{Deterding}.
StackOverflow is a popular Q&A website for an ever-increasing range of computer programming topics. SO is likely to be the most popular website in the computer programming world with over 4,000,000 registered users and with more than 11,000,000 questions. The website serves as a platform for users to both ask questions related to programming and to answer the questions of other users. These contributions by the users are the heart of the content and are met with reputation points and “badges” for the user. What we hope to uncover is whether this use of gamification is the primary motivating factor for the user’s collaboration and if so, is it possible to improve the current system in order to foster an increase in collaboration by users that wouldn’t normally contribute?
This research aims to examine the ways that collaboration, in the form of asking and answering questions and commenting on the questions, is achieved in SO and what changes can be made to encourage this kind of collaboration from more users and to promote an increase in contributions from users of all skill levels.
Throughout the course of this research, we will present an overview of relevant, related work, examine what the primary motivating factors are for user collaboration, discuss the merits and shortcomings of the current system, and provide our recommendations of how StackOverflow may be improved to encourage more active participation from a wider range of its users.

\section{Related work}
StackOverflow has been heavily studied in recent years. The following papers
are the most representative and share valuable insights into our own research:

\subsection{StackOverflow and GitHub}

Associations between Software Development and Crowdsourced Knowledge which
successfully linked users from the two platforms and determined that “for
active committers, asking questions on StackOverflow catalyzes committing on
GitHub and similarly, for active committers, answering questions on
StackOverflow catalyzes committing on GitHub.”

\subsection{One-day flies on StackOverflow}

Why the vast majority of StackOverflow users only post once which found that
“less active users” are more likely to not have questions answered, get
negative feedback (or answers that can be interpreted that way), self-answer
their own question and less likely to have questions not already answered.

\subsection{Can Gamification Motivate Voluntary Contributions? The case of
StackOverflow Q\&A Community}

This research was focused on answering activity considering a dataset with
46,571 StackOverflow users. The analysis was focused on comparing the answering
activity 7 days before getting the badge and 7 days after the awarding date.
The results suggested a positive impact on increasing the number of answers due
to badges. For example, those users increasing reputation when answering
questions are more likely to answer more questions as a result of the
gamification effect. The analysis was more focused on the behavior of members after
receiving a badge. In general, the research has shown that badges motivate
the participation of the user. However, we argue that a change in the current
classification of badges besides the inclusion of social media could boost the
increase of participation in StackOverflow.

\subsection{Does Gamification Work?}

This research indicates that gamification provides positive effects, however,
the effects are greatly dependent on the context in which the gamification is
being implemented, as well as on the users using it. Most of the reviewed
papers reviewed positive results for the same of the motivational affordances
of the gamifications implementations studied.

\subsection{Building Reputation in StackOverflow}

The authors analyzed the StackOverflow (SO) data from four perspectives to
understand the dynamics of reputation building on SO. They found that a large
number of questions are related to .NET technologies, OOP languages and web
development. Therefore contributors with expertise in those topics will have
greater chance building reputation quickly. A contributor should participate
regularly and answer as many questions as possible. These actions will improve
the contributor’s influence and chances of getting up-votes.

\subsection{StackOverflow Badges and User Behavior: An Econometric Approach}

The authors included the different actions to increase reputation, such as
editing posts, asking questions and answering questions. They discovered that
the behavior changes according to the action to obtain the badge. For example,
users earning a badge for editing tend to make more edits in the prior days to
receiving a badge compared to days after. In addition, the users who receive a
badge for asking questions are not motivated to obtain more badges. The authors
proposed anonymous questions feature to increase the rate of posted questions.
However, a system of anonymous questions can decrease the level of quality and
increase the number of questions.


\section{Research Methodology}
To study the collaboration in StackOverflow through answering questions,
qualitative research will allow us to address human aspects such as motivation
to increase the contributions on the community. On StackOverflow, the
contributions are composed of questions and answers and the up/downvoting of
other users answers. However, this research is
focused on maximizing the contributions on questions through proposing a more
detailed classification in the elements of gamifications and social media as an
external factor to strengthen the current gamification. The next sections are
dedicated to the research questions and study design to be followed.

\subsection{Research Questions}

subsubsection{Is the publication of reputation and badges on social media a
factor to promote more answers?}

This question will explore the user reaction when sharing reputation and badges
on social media such as Facebook, Twitter, Linked-in, etc. We will study if this
additional recognition can stimulate the contributions in the community.


subsubsection{Is the current gamification used by StackOverflow meeting the
user requirements?}

This question will provide us the possible dissatisfaction in the user
experience related to gamification and find the possible improvements on
current classification of badges. The result will involve a change on how the
badges are being classified to more suitable classification including more
detailed characteristics such as programming languages and other areas of expertise.

subsubsection{How would the users like to include their reputation and badges
on social media?}

This question will provide the user’s vision about the specific information
that they would like to share on social media. Based on the second question,
users can provide recommendations on how they would like to export their
achievements and which information they would consider important to be
included and what is ignored.


\subsection{Study Design}

This research will include two groups of StackOverflow with 30 participants
each one. In the first group, the members are retrieved by the data dump
available in StackOverflow website and considering three different levels of
reputation: low, medium and high level. A second group will consist of UVic
students who are currently taking CSCW subject and use StackOverflow. In the
first phase, we will send Surveys to obtain information about the current
gamification and possible impact of social media. In the second phase,
unstructured interviews through Skype will be applied to participants of the
first group. In addition, we will apply interviews to the second group in
person. The insight of having two different groups would provide us some
differences in the answers due to age and profile. In the third phase, we will
apply survey questionnaires including changes in the user interface related to
gamification and collect again the answers to confirm the possible impact of
social media and gamification on the community.


\section{Risks}

Part of the results of this project depends on the availability of the SO
members chosen from the data dump offered by SO. Not having the insights of
these users might reduce the domain range and levels of reputation in the
study. We will consider to automatize a process to crawl the email users and
send massive emails to obtain the needed participation. In addition, the
recommendations to improve Gamification might lead to having the SO community
being biased and sarcastic. Finally, given the limited time, we might carry out
interviews and surveys for small domain of users.


\section{Expected Results}

After collecting all the data from our research we expect to have a better
understanding of how gamification and social media affects StackOverflow and its
users. To be more precise, we will look at these game design elements such as
badges and reputation points. We will be collecting data to determine  whether
or not the current gamification is working to prompt users to be more active
and engaging. Based on our findings we will hope to add our own contributions
which will improve this system. We will also be looking at how social media is
impacting the user experience. Our expected contributions will be based on the
data we collect and the answers to our research questions. We are looking to
end this extensive study with concrete improvements to StackOverflow’s:
gamification implementation, social media synergy and user interaction.




% REFERENCES FORMAT
% References must be the same font size as other body text.
% \bibliographystyle{SIGCHI-Reference-Format}
% \bibliography{document}

\begin{thebibliography}{9}

\bibitem{Antin}
Antin, Judd, and Elizabeth F. Churchill. "Badges in social media: A social psychological perspective." CHI 2011 Gamification Workshop Proceedings (Vancouver, BC, Canada, 2011). 2011.

\bibitem{Bosu}
Bosu, Amiangshu, et al. "Building reputation in stackoverflow: an empirical investigation." Proceedings of the 10th Working Conference on Mining Software Repositories. IEEE Press, 2013.

\bibitem{Caponetto}
Caponetto, Ilaria, Jeffrey Earp, and Michela Ott. "Gamification and Education: A Literature Review." ECGBL2014-8th European Conference on Games Based Learning: ECGBL2014. Academic Conferences and Publishing International, 2014.

\bibitem{Cavusoglu}
Cavusoglu, Huseyin, Zhuolun Li, and Ke-Wei Huang. "Can Gamification Motivate Voluntary Contributions?: The Case of StackOverflow Q\&A Community." Proceedings of the 18th ACM Conference Companion on Computer Supported Cooperative Work \& Social Computing. ACM, 2015.

\bibitem{Deterding}
Deterding, Sebastian, et al. "Gamification. Using game-design elements in non-gaming contexts." CHI'11 Extended Abstracts on Human Factors in Computing Systems. ACM, 2011.

\bibitem{Jin}
Jin, Yong, et al. "Quick Trigger on Stack Overflow: A Study of Gamification-influenced Member Tendencies." (1916).

\bibitem{Marder}
Marder, Andrew. "Stack overflow badges and user behavior: an econometric approach." Proceedings of the 12th Working Conference on Mining Software Repositories. IEEE Press, 2015.

\bibitem{McGonigal}
McGonigal, Jane. Reality is broken: Why games make us better and how they can change the world. Penguin, 2011.

\bibitem{Hägglund}
Hägglund, Per. "Taking gamification to the next level." (2012).

\bibitem{Hamari}
Hamari, Juho, Jonna Koivisto, and Harri Sarsa. "Does gamification work?--a literature review of empirical studies on gamification." System Sciences (HICSS), 2014 47th Hawaii International Conference on. IEEE, 2014.

\bibitem{Rughinis}
Rughinis, Razvan. "Gamification for productive interaction: Reading and working with the gamification debate in education." Information Systems and Technologies (CISTI), 2013 8th Iberian Conference on. IEEE, 2013.

\bibitem{Ryan}
Ryan, Richard M., C. Scott Rigby, and Andrew Przybylski. "The motivational pull of video games: A self-determination theory approach." Motivation and emotion 30.4 (2006): 344-360.

\bibitem{Slag}
Slag, Rogier, Mike de Waard, and Alberto Bacchelli. "One-day flies on StackOverflow."

\bibitem{Vasilescu}
Vasilescu, Bogdan, Vladimir Filkov, and Alexander Serebrenik. "StackOverflow and GitHub: associations between software development and crowdsourced knowledge." Social Computing (SocialCom), 2013 International Conference on. IEEE, 2013.

\bibitem{Wang}
Wang, Shaowei, David Lo, and Lingxiao Jiang. "An empirical study on developer interactions in StackOverflow." Proceedings of the 28th Annual ACM Symposium on Applied Computing. ACM, 2013.


\end{thebibliography}

\end{document}

%%% Local Variables:
%%% mode: latex
%%% TeX-master: t
%%% End:
